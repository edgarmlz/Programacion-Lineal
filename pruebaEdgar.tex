\documentclass{article}
\usepackage[utf8]{inputenc}
\usepackage{amsmath}
\usepackage[spanish]{babel}
\title{Apuntes de Programación lineal}
\author{Reyna Edgar}

\begin{document}

\maketitle
\tableofcontents
\section{Introducción}
\label{sec:introduccion}
La forma estándar de un problema de programación lineal es:
Dada una matriz $A$ y vectores $B$ y $C$ maximizar $C^TX$ sujeto a
$AX\leq b$.\newline
La forma simplex de un problema de programación lineal es:
Dada una matriz $A$ y vectores $B$ y $C$ maximizar $C^TX$ sujeto a
$AX= b$.

\section{Ejemplo de poner una tabla}
\label{sec:ejemplo-de-poner}
\begin{tabular}{|c|c|c|}
  \hline
           &a&b\\
  \hline
  maquina 1&1&2\\
  maquina 2&1&1\\
  \hline
\end{tabular}

\section{Ejemplo de poner una matrix}
\label{sec:poner-tablas}

\begin{equation}
  \label{eq:1}
  A=
  \begin{pmatrix}
    0&1&2\\
    3&1&5
  \end{pmatrix}
  *
  \begin{pmatrix}
    0&1&2\\
    1&2&3
  \end{pmatrix}
\end{equation}

\end{document}


